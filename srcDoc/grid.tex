\section{Setting the Grid}
\label{grid.sec}

Setting the grid resolution in GITM is not very complicated, but it
does involve some thought.  There are a few variables that control
this.  In {\tt ModSize.f90}, the following variables are defined:

\begin{verbatim}
integer, parameter :: nLons = 9
integer, parameter :: nLats = 9
integer, parameter :: nAlts = 50

integer, parameter :: nBlocksMax = 4
\end{verbatim}

The first three variables (nLons, nLats and nAlts) define the size of a single block.  In the example above, there are 9 cells in latitude, 9 cells in longitude and 50 cells in altitude.  The latitude and longitude resolution in GITM is defined by the cell numbers specified here and the block numbers discussed in the following section.  The size of the altitude cells are measured in scale heights instead of kilometers, as certain regions require higher resolutions than others based on the dominating chemical and dynamical processes.  Each altitude cell contains $\frac{1}{3}$ of a scale height, starting at 80 km and typically reaching up to 500 km.  Increasing the number of altitude blocks will increase the altitude range, but if this value is increased too much the model becomes unstable.  This altitude limit makes it problematic to model the equatorial region during storms, where increased vertical plasma transport requires the model consider altitudes of 2000 km and higher.

The final variable (nBlocksMax) defines the maximum number of blocks you can have on a single processor.  Most people run with one single block per processor, as this is faster.  So setting this to ``1'' is usually preferred, and is necessary when running with the Dynamo option on.  This is a necessity because the Dynamo routine does not correctly transfer the geomagnetically oriented data into the geographic coordinates used in the ghost cells.  Hopefully this will be updated in future versions, as allowing multiple blocks per node can, in theory, save memory.

Don't forget, if you change any of these parameters you will need to recompile the code (run the Makefile again) so that a new executable using the newly defined variables can be produced.

\subsection{Running 3D Over the Whole Globe}

Once the number of cells is defined, then the number of blocks in latitude and longitude need to be defined.  This is done in the {\tt UAM.in} file, as shown in section~\ref{uam.sec}.  For example, the initial settings have 8 blocks in latitude and 8 in longitude:

\begin{verbatim}
#GRID
8           lons
8           lats
-90.0       minimum latitude to model
90.0        maximum latitude to model
0.0         minimum longitude to model
0.0         maximum longitude to model
\end{verbatim}

The number of cells in the simulation domain will then be 72 in longitude, 72 in latitude, and 50 in altitude.  Given that there are 360$\deg$ in longitude and $180\deg$ in latitude, the resolution would be $360\deg/72 = 5.0\deg$ and $180\deg/72 = 2.5\deg$ in latitude.  If one block were put on each processor, 64 processors would be required.

If one desired to run without the Dynamo turned on, the problem could also be run with multiple blocks per node.  This grid fits quite nicely on either four- or eight-core processors.  However, on 12-core processors this would not work very well at all.  We have started to run simulations of $1\deg$ in latitude and $5\deg$ in longitude, which can fit nicely on a 12-core processor machine.  For example, in {\tt ModSize.f90}:

\begin{verbatim}
integer, parameter :: nLons = 9
integer, parameter :: nLats = 15
\end{verbatim}

and in {\tt UAM.in}:

\begin{verbatim}
#GRID
8		lons
12		lats
\end{verbatim}

This can then run on 96 cores, which is nicely divisible by 12.  Essentially, an infinite combination of cells per block and number of blocks can be utilized.  Typically, the number of blocks in latitude and longitude are even numbers.

\subsection{Running 3D Over the Part of the Globe}

GITM can be run over part of the globe - both in latitude and in
longitude.  It can be run over a single polar region (by setting
either the minimum or maximum latitude to be greater (or less) than
$\pm 90\deg$).  If this is selected, message passing over the poles is
implemented.  If the pole is not selected, then boundary conditions
have to be set in {\tt set\_horizontal\_bcs.f90}.  By default, a
continuous gradient boundary condition is used on the densities and
temperatures, while a continuous value is used on the velocity.  This
is true in both latitude and longitude.  In longitude, message passing
is implemented all of the time, but the values are over-written by the
boundary conditions if the maximum and minimum longitude are not equal
to each other.

The longitudinal resolution ($\Delta{\phi}$) is set by:
\begin{equation}
\Delta{\phi} = \frac{\phi_{end} - \phi_{start}}{nBlocksLon \times nCellsLon}
\end{equation}
while, the latitudinal resolution ($\Delta{\theta}$) is set by:
\begin{equation}
\Delta{\theta} = \frac{\theta_{end} - \theta_{start}}{nBlocksLat \times nCellsLat}
\end{equation}

\subsection{Running in 1D}

GITM can run in 1D mode, in which the call to advance\_horizontal is
not completed.  This means that GITM runs exactly the same way, but
ignoring all of the horizontal advection terms.  You have to do two
things to make GITM run in 1D.  First, in {\tt ModSize.f90}:
\begin{verbatim}
integer, parameter :: nLons = 1
integer, parameter :: nLats = 1
integer, parameter :: nAlts = 50

integer, parameter :: nBlocksMax = 1
\end{verbatim}
This tells the code that you only want one single latitude and
longitude location.  To specify the exact location, in {\tt UAM.in}:
\begin{verbatim}
#GRID
1           lons
1           lats
41.75       minimum latitude to model
41.75       maximum latitude to model
275.0       minimum longitude to model
275.0       maximum longitude to model
\end{verbatim}
This is pretty close to some place in Michigan.  GITM will model this
exact point for as long as you specify.  One thing to keep in mind
with running in 1D is that the Earth still rotates, so the spot will
have a strong day to night variation in temperature.  In 3D, the winds
decrease some of the variability between day and night, but in 1D,
this doesn't happen.  So, the results are going to be perfect.  But,
1D is great for debugging.
